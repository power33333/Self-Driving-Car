\documentclass{article}
\usepackage[utf8]{inputenc}
\usepackage{amsmath}
\usepackage{amsfonts}

\title{Mathematical formulation of the obstacle-free problem}
\author{}
\date{November 2022}

\begin{document}

\maketitle

\section{Problem formulation}
Here, we present the single track model of the car.

\paragraph{Notation}

\begin{itemize}
    \item \(v\): velocity of car
    \item \(v_f\): velocity of front wheel
    \item \(v_r\): velocity of rear wheel
    \item \(\delta\): steering angle
    \item \(\beta\): side slip angle
    \item \(\psi\): yaw angle
    \item \(\alpha_f\): side slip angle at front wheel
    \item \(alpha_r\): side slip angle at rear wheel
    \item \(F_sf\): lateral tire force at front wheel
    \item \(F_sr\): lateral tire force at rear wheel
    \item \(F_lf\): longitudnal tire force at front wheel
    \item \(F_lr\): longitudnal tire force at rear wheel
    \item \(l_f\): distance from centre of gravity to front wheel centre
    \item \(l_r\): distance from centre of gravity to rear wheel centre
    \item \(e_{SP}\): distance from centre of gravity to drag mount point 
    \item \(F_{Ax}\): air resistance in x direction in vehicle frame
    \item \(F_{Ay}\): air resistance in y direction in vehicle frame
    \item \(m\): mass of the car
\end{itemize}

\paragraph{Control variables and constraints}
\begin{itemize}
    \item steering angle velocity \( w_{\delta} \leq 0.5 \) 
    \item braking force \( 0 \leq F_B \leq 15000 \) [N]
    \item gas pedal position \( \phi \in [0,1]\)
\end{itemize}

\paragraph{Diagram}
Draw diagram here

\paragraph{Equations of motion}

Anticlockwise direction is assumed as positive direction of rotation. Therefore angles \(\beta, \alhpha_f, \alpha_r \leq 0 \)

In the world coordinate system, 
\begin{align}
    \dot{x} = v cos(\psi - \beta) \\
    \dot{y} = v sin(\psi - \beta)
\end{align}
\\

Balancing forces along the direction of velocity,
\begin{align}
    \dot{v} = \dfrac{1}{m} \bigg[ (F_{lr} - F_{Ax})cos\beta - (F_{sr} - F_{Ay})sin\beta + F_{lf}cos(\delta + \beta) - F_{sf}sin(\delta + \beta)
    \bigg]
\end{align}
\\

Balancing the centrifugal forces on the COG,
\begin{align}
    \dot{\beta} = \omega_z - \dfrac{1}{mv}\bigg[ (F_{lr} - F_{Ax})sin\beta + (F_{sr} - F_{Ay})cos\beta + F_{lf}sin(\delta + \beta) + F_{sf}cos(\delta + \beta)
    \bigg]
\end{align}
\\

Yaw rate is the angular velocity around the z-axis on COG,
\begin{align}
    \dot{\psi} = \omega_z
\end{align}

Using conservation of angular momentum on COG,
\begin{align}
    \dot{\omega_z} = \dfrac{1}{I_{zz}}\bigg[ F_{sf}.l_f.cos\delta - F_{sr}.l_r - F_{Ay}.e_{SP} + F_{sf}.l_f.sin\delta \bigg]
\end{align}

Lastly, the steering angle velocity,
\begin{align}
    \dot{\delta} = \omega_{\delta}
\end{align}

\paragraph{Tire models}

We use the non-linear tire model of Pacejka magic formula:

\begin{align}
    F_{sf}(\alpha_f) = D_f.sin(C_f arctan(B_f\alpha_f - E_f(B_f\alpha_f - arctan(B_f\alpha_f))))\\
     F_{sr}(\alpha_r) = D_r.sin(C_r arctan(B_r\alpha_r - E_r(B_r\alpha_r - arctan(B_r\alpha_r))))
\end{align}
\\

Using the translational velocity component of v on the rear and front wheels, the side slip angles at the front and rear wheels are given by

\begin{align}
    \alpha_f = \delta - arctan \bigg( \dfrac{l_f. \dot{\psi} - vsin\beta}{vcos\beta} \bigg)\\
     \alpha_r = arctan \bigg( \dfrac{l_r. \dot{\psi} + vsin\beta}{vcos\beta} \bigg)
\end{align}

\paragraph{Air drag}

The air drag along the lateral direction of the vehicle is assumed to be zero and along the longitudnal direction it is given by
\begin{align}
    F_{Ax} = \dfrac{1}{2} c_{\omega} \rho A v^2\\
    F_{Ay} = 0
\end{align}

We assume a front wheel drive and a constant gears.
The longitudnal tire forces are related to the breaking forces and the rolling resistances at the tires by,

\begin{align}
    F_{lf} = \dfrac{M_{wheel}(\phi)}{R} - F_{Bf} - F_{Rf}\\
    F_{lr} = -F_{Br} - F_{Rr}
\end{align}

The motor torque at the front wheel \(M_wheel\) is given by

\begin{align}
    M_{wheel}(\phi) = i_g. i_t. f_1(\phi). f_2(\omega_{mot}(\phi)) + (1-f_1(\phi))f_3(\omega_{mot}(\mu))\\
    \omega_{mot} = \dfrac{v.i_g.i_t}{R}
\end{align}

The three functions are given by 
\begin{align}
    f_1(x) = 1 - exp(-3x)\\
    f_2(x) = -37.8 + 1.54x-0.0019x^2\\
    f_3(x) = -34.9 - 0.04775x
\end{align}

We assume that the breaking force \(F_B\) is distributed as

\begin{align}
    F_{Bf} = \dfrac{2}{3}F_B, F_{Br} = \dfrac{1}{3}F_B
\end{align}

The rolling resistance forces are given by,

\begin{align}
    F_{Rf} = f_R(v).F_{zf}, F_{Rr} = f_R(v).F_{zr}\\
    f_R(v) = f_{R0} + f_{R1}\dfrac{v}{100} + f_{R4}\bigg(\dfrac{v}{100}\bigg)^4\\
    F_{zf} = \dfrac{m.l_r.g}{l_f + l_r}\\
    F_{zr} = \dfrac{m.l_f.g}{l_f + l_r}
\end{align}

(23) is the friction coefficient, and (24) and (25) give the static tire loads

\paragraph{Road constraints}
We assume a uniform width \( r_w\) of the road and therefore the y state has to be constrained by 
\begin{align}
    r_w + B/2 \leq y(t) \leq r_w - B/2
\end{align}
where B is the width of the car.

\paragraph{Initial conditions}
\begin{align}
    (x(0), y(0), v(0), \beta(0), \psi(0), \omega_z(0), \delta(0)) = 
    (0, free, 0, 0, 0, 0, 0)
\end{align}

\paragraph{Final conditions}

The time taken to complete the course is given by \(t_f\) which is a free parameter

\begin{align}
    x(t_f) = 140, \psi(tf) = 0
\end{align}

\paragraph{Loss function}

The constrained optimisation problem aims to minimise the time taken to complete the course and the total steering effort

Minimise
\begin{equation}
    t_f + \int_0^{t_f} w_\delta(t)^2 dt
\end{equation}

with constraints
\begin{align}
    w_\delta(t) \in [-0.5, 0.5], F_B(t) \in [0,150000], \psi \in [0,1]
\end{align}


\section{Converting to a finite-dimensional optimization problem}

\subsection{Define state, control and constraints}

Define functions for state z, control u, and functions f, g, \(\phi\), \(s\) for the dynamics, final conditions, state constraints, 

\begin{align*}
    & z: t \xrightarrow{} \mathbb{R}^7,  z(t) = (x,y,v,\beta, \psi, \omega_z, \delta)^T(t)\\
    & u: t \xrightarrow{} \mathbb{R}^3, u(t) = (w_\delta, F_B, \phi)^T(t)\\
\end{align*}

dynamics of z is described by, \(f: \mathbb{R} \times \mathbb{R}^7 \times \mathbb{R}^3\)
\begin{align}
    \dot{z}(t) &= f(t,z(t),u(t))\\ & := \begin{bmatrix}
                            v cos(\psi - \beta) \\
                             v sin(\psi - \beta) \\
                             \dfrac{1}{m} \bigg[ (F_{lr} - F_{Ax})cos\beta - (F_{sr} - F_{Ay})sin\beta + F_{lf}cos(\delta + \beta) - F_{sf}sin(\delta + \beta)\\
    \bigg] \\
    \omega_z - \dfrac{1}{mv}\bigg[ (F_{lr} - F_{Ax})sin\beta + (F_{sr} - F_{Ay})cos\beta + F_{lf}sin(\delta + \beta) + F_{sf}cos(\delta + \beta)
    \bigg] \\
                            \omega_z \\
                            \dfrac{1}{I_{zz}}\bigg[ F_{sf}.l_f.cos\delta - F_{sr}.l_r - F_{Ay}.e_{SP} + F_{sf}.l_f.sin\delta \bigg]\\
                              \omega_\delta
            \end{bmatrix}
\end{align}


final and initial conditions,
\begin{align}
    \gamma(z(t_0), z(t_f)) := \big(x(t_0), x(t_f)-x_{END})^T = (0, 0)^T
\end{align}

control constraints,
\begin{align}
    c(t, z(t), u(t)) &:= \begin{bmatrix}
                            w_\delta(t) - 0.5 \\
                          -0.5 - w_\delta(t) \\
                           -F_B(t) \\
                           F_B(t) - 15000\\
                           -\phi(t) \\
                           \phi(t) - 1\\
         \end{bmatrix} \leq 0_{\mathbb{R}^8}
\end{align}

state constraints,
\begin{align}
    s(t, z(t)) &:= \begin{bmatrix}
                            y(t) - y_{up}(x(t)) \\
                            y_{down}(x(t)) - y(t)\\
                    \end{bmatrix} \leq 0_{\mathbb{R}^2}
\end{align}

where \(y_{up}, y_{down}\) are functions describing road boundary y-coordinates



\subsection{Converting to a fixed time problem}

Assuming \(t_0 = 0\), the final time \(t_f\) is free so we use the linear time transformation
\begin{equation}
    t(\tau) = \dfrac{\tau}{t_f},  \tau \in [0,1]
\end{equation}

and introduce the state \(t_f\) which is constant for \(\tau \in [0,1]\) and \( \dfrac{d}{d\tau}t_f(\tau) = 0 \)

For state z, define 
\begin{align*}
\bar{z}(\tau) := z(t(\tau)) = z(\tau t_f) \\
\bar{u}(\tau) := u(t(\tau)) = u(\tau t_f)
\end{align*}
Then, 

\begin{align*}
    \dfrac{d}{d\tau}\bar{z}(\tau) &= \dot{z}(t(\tau)). \dfrac{d}{d\tau}({t(\tau))}\\
    & = f(t(\tau), z(t(\tau)). , u(t(\tau))). \dfrac{1}{t_f} \\
    &= \dfrac{1}{t_f(\tau)}f\big(\dfrac{\tau}{t_f}, \bar{z}(\tau), \bar{u}(\tau)\big) 
\end{align*}

Now we redefine \(\bar{z}(\tau) = (\bar{z}(\tau), t_f(\tau))\) to include the new state.\\

Then the optimization problem is given by,\\

Minimize
\begin{align*}
    t_f(\tau)
\end{align*}
 
with respect to \(\bar{z}\) and \(\bar{u}\),
subject to the ODEs,

\begin{align*}
    \dot{\bar{z}}(t) &= f\bigg(\frac{\tau}{t_f},\bar{z}(t),\bar{u}(t))\bigg)\
\end{align*}

and constraints,

\begin{align*}
    \bar{c} \bigg(\dfrac{\tau}{t_f},\bar{z}(t),\bar{u}(t)\bigg) := c(t(\tau), z(t(\tau)) , u(t(\tau))) \leq 0_{\mathbb{R}^8}\\
    \bar{s} \bigg(\dfrac{\tau}{t_f},\bar{z}(t),\bar{u}(t)\bigg) := s(t(\tau), z(t(\tau)) , u(t(\tau))) \leq 0_{\mathbb{R}^2}\\
    \gamma(\bar{z}_0, \bar{z}_N) = \big(\bar{x}(\tau_0), \bar{x}({\tau_N})-x_{END})^T = (0, 0)^T
\end{align*}

\subsection{Discretization}
We define the grid with N steps,

\begin{align*}
G_N = [\tau_0 = 0, \tau_1, \tau_2, \dots, \tau_N = 1]
\end{align*}

\subsubsection{Discretization of control space}
We discretize the control space with the coefficients \( m = (m_1, m_2, \dots, m_M)^T \) and B-splines of order 3, \( B_1, B_2, \dots, B_{N+2}\) given by

\begin{align}    
B_i^3(\tau) = 
\begin{cases}
      \dfrac{(\tau - T_i)^2}{ (T_{i+2}-T_i)(T_{i+1} - T_i)} & \text{if } \tau \in [T_i, T_{i+1})\\\\
      \dfrac{(\tau - T_i)(T_{i+2}-\tau)}{(T_{i+2}-T_i)(T_{i+2} - T_{i+1})} + 
      \dfrac{(T_{i+3}-\tau)(\tau-T_{i+1})}{(T_{i+3}-T_{i+1})(T_{i+2} - T_{i+1})} & \text{if } \tau \in [T_{i+1}, T_{i+2}) \\\\
     \\\\
      \dfrac{(T_{i+3} - \tau)^2}{ (T_{i+3}-T_{i+1})(T_{i+3} - T_{i+2})} & \text{if } \tau \in [T_{i+2}, T_{i+3}) \\\\
      0, & \text{otherwise}
\end{cases}
\end{align}

Here \( M = 3(N+3-1) = 3(N+2) \)

The discretised control space is given by,
\begin{equation}
    U^M := \left\{ \sum_{i=1}^{N+2} m_i. B_i^3 | m_i \in \mathbb{R}^3, i =1,2, \dots, N+2 \right\}
\end{equation}


\subsubsection{Full discretization}

Then we use Heun's method to discretize the ODE, with increment function \(\kappa\) given by
\begin{align*}
    \kappa(\tau, \bar{z}, m, h) = \dfrac{1}{2}(f(\tau, \bar{z}, \bar{u}_M) + f(\tau + h, \bar{z} + hf(\tau, \bar{z}, \bar{u}_M), \bar{u}_M)),\\
    \text{where} \; \bar{u}_M(\tau) = \sum_{i=1}^M m_i. B_i^3 (\tau)
\end{align*}


Then the discretized optimisation problem is, 

Minimize
\begin{align*}
    t_f(\tau_N)
\end{align*}
 
with respect to \(\bar{z}_N : G_N \xrightarrow{} \mathbb{R}^8\)  and \(\bar{u_N}: G_N \xrightarrow{} \mathbb{R}^3\),
subject to,

\begin{align*}
    \bar{z}_j + h_j. \kappa(\tau_j,\bar{z}_j,m, h_j)) - \bar{z}_{j+1} = 0_{\mathbb{R}^8}, \; j =0,1,\dots, N-1 \
\end{align*}

and constraints,

\begin{align*}
    \bar{c}(\tau_j,\bar{z}_j,\bar{u}_M(\tau_j; m)) \leq 0_{\mathbb{R}^8}, \; j =0,1,\dots, N-1  \\
    \bar{s} (\tau_j,\bar{z}_j) \leq 0_{\mathbb{R}^2}\; j =0,1,\dots, N-1  \\
    \gamma(\bar{z}_0, \bar{z}_N) = \big(\bar{x}(\tau_0), \bar{x}({\tau_N})-x_{END})^T = (0, 0)^T
\end{align*}

Collecting all the variables and constraints into matrices, we get the optimization problem,\\

Minimise

\begin{align*}
    J(\bar{q}) = t_f(\tau_N)
\end{align*}

with respect to \(\bar{q} = (\bar{z}_1, \bar{z}_2, \dots, \bar{z}_N, m) \in \mathbb{R}^{8(N+1)+3} \)

with constraints

\begin{equation}
    G(\bar{q}) & = \begin{bmatrix}
                    c(\tau_0, \bar{z}_0, \bar{u}_M(\tau_0; m))\\
                    c(\tau_1, \bar{z}_1, \bar{u}_M(\tau_1; m))\\
                    \vdots\\
                    c(\tau_N, \bar{z}_N, \bar{u}_M(\tau_N; m))\\
                    s(\tau_0, \bar{z}_0)\\
                    s(\tau_1, \bar{z}_N)\\
                    \vdots\\
                    s(\tau_N, \bar{z}_N)
                \end{bmatrix} \leq 0_{\mathbb{R}^{(8+2)(N+1)}}
\end{equation}

and,
\begin{equation}
    H(\bar{q}) & = \begin{bmatrix}
                     \bar{z}_0 + h_0. \kappa(\tau_0,\bar{z}_0,m, h_0)) - \bar{z}_1\\
                    \bar{z}_1 + h_1. \kappa(\tau_1,\bar{z}_1,m, h_1)) - \bar{z}_2\\
                    \vdots\\
                    \bar{z}_{N-1} + h_{N-1}. \kappa(\tau_{N-1},\bar{z}_{N-1},m, h_{N-1})) \bar{z}_N\\
                    \gamma(\bar{z}_0, \bar{z}_N)
                \end{bmatrix} = 0_{\mathbb{R}^{8(N+1) + 2}}
\end{equation}


\end{document}

